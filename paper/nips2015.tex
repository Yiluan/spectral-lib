\documentclass{article} % For LaTeX2e
\usepackage{nips15submit_e,times}
\usepackage{hyperref}
\usepackage{url}
\usepackage{amsmath}
\usepackage{amssymb}
\usepackage{amsthm}
\usepackage{algpseudocode}
\usepackage{algorithm}
%\documentstyle[nips14submit_09,times,art10]{article} % For LaTeX 2.09

\newtheorem{graphconv}{Definition}


\title{Deep Networks on Graph-Structured Data}


\author{
Mikael Henaff \\
Courant Institute of Mathematical Sciences\\
New York University\\
\texttt{mbh305@nyu.edu} \\
\And
Joan Bruna \\
University of California, Berkeley \\
\texttt{joan.bruna@berkeley.edu} \\
\AND
Yann LeCun \\
Courant Institute of Mathematical Sciences \\
New York University \\
\texttt{yann@cs.nyu.edu} \\
}

% The \author macro works with any number of authors. There are two commands
% used to separate the names and addresses of multiple authors: \And and \AND.
%
% Using \And between authors leaves it to \LaTeX{} to determine where to break
% the lines. Using \AND forces a linebreak at that point. So, if \LaTeX{}
% puts 3 of 4 authors names on the first line, and the last on the second
% line, try using \AND instead of \And before the third author name.

\newcommand{\fix}{\marginpar{FIX}}
\newcommand{\new}{\marginpar{NEW}}

%\nipsfinalcopy % Uncomment for camera-ready version

\begin{document}


\maketitle

\begin{abstract}

\end{abstract}

\section{Introduction}

%main ideas here:
% how to perform weight sharing in non-euclidean domains. 
% how to estimate a graph structure 
% towards efficient archiectures

In recent times, deep learning models have proven extremely successful in a wide variety of tasks, from computer vision and acoustic modeling, to natural language processing \cite{natureyann}. At the core of their success lies an important assumption on the statistical properties of the data, namely the \emph{stationarity} and the \emph{compositionality} through local statistics, which is present in natural images, video, or speech.
 These properties are exploited efficiently by ConvNets ~\cite{krizhevsky2012,hinton12}, which are designed to extract local features that are shared across the signal domain. Thanks to this, they are able to greatly reduce the number of parameters in the network with respect to generic Deep architectures, without sacrificing the capacity to extract informative statistics from the data. Similarly, Recurrent Neural Nets (RNNs) trained on temporal data implicitly assume a stationary distribution.

%
%
%%The 
%%Thus far ConvNets have proved extremely successful for a large variety of tasks in computer vision and acoustic modeling . This is largely due to their ability to efficiently exploit stationarity and local statistics to   
%%As a result, one is able to train very large networks while limiting the overfitting problem.
%When the data lives on a low-dimensional grid 
%(1-D for speech or text data, 2-D for images, 3-D for videos), a ConvNet can exploit 
% 
%Images can be thought of as signals defined on a 2-D lattice, where each pixel is a node which is connected to its immediate neighbors. 
%Similarly, audio waveforms can be viewed as signals defined on a 1-D lattice, where each node is a time point. 
%Generalizing convolutional networks to graph-structured data is an important open problem. 
One can think such data as being signals defined on a low-dimensional grid, where the stationarity can be well defined via the natural translation 
operator on the grid, locality is defined via the metric of the grid, and compositionality is obtained from downsampling. 
However, many types of data are defined on more complex graphs. For example, text documents represented as bags of words can be thought of as signals defined on a graph whose nodes are vocabulary terms and whose weights represent some similarity measure between terms, such as co-occurence statistics. In medicine, a patient's gene expression data can be viewed as a signal defined on the graph imposed by the regulatory network. In fact, computer vision and audio, which are the main focus of research efforts in deep learning, only represent a special case of data defined on an extremely simple low-dimensional graph. For those type of data of dimension $n$, deep learning strategies are reduced to learning with fully-connected layers, which have $o(n^2)$ parameters, and regularization is carried out via weight decay and dropout \cite{dropouthinton}.

When the graph structure of the input is known, \cite{spectralnet2013} introduced a model to generalize convnets using low learning complexity, that was shown on simple low-dimensional graphs. In this work, we are interested in generalizing ConvNets to more general data distributions, and, most importantly, to the setting where the graph structure is not known a priori. In this context, learning the graph structure amounts to estimating the similarity matrix, which has complexity $o(n^2)$. One may therefore wonder whether the graph estimation followed by graph convolutions offers advantages with respect to learning directly from the data with fully connected layers. We attempt to answer this question experimentally as well as mathematically. 
 
We explore these approaches in two areas of application for which it has not been possible to apply convolutional networks before: text categorization and bioinformatics. Our results show that our method is capable of matching or outperforming large, fully-connected networks trained with dropout using fewer parameters. 
%Main contributions of this work. How to apply graph convolutions on problems where the graph is not known a priori.
Our main contributions can be summarized as follows:
\begin{itemize}
\item We extend the ideas from \cite{spectralnet2013} to large scale classification problems, on Object Recognition, text categorization and bioinformatics.
\item We consider the most general setting where no prior information on the graph structure is available, and propose unsupervised and supervised graph estimation strategies.
\item Finally, we introduce an alternative formulation for efficient learning in graph-structured domains which works directly in the feature domain.
\end{itemize}

The rest of the paper is structured as follows. Section \ref{relatedworksect} reviews similar works in the literature. Section \ref{spectralsect} discusses generalizations of convolutions on graphs, and Section \ref{graphestimsect} addresses the question of graph estimation. Finally, Section \ref{experimentssect} shows numerical experiments on large scale object recogniton, text categorization and bioinformatics.

\section{Related Work}

Spectral Net

Locally connected from Coates et Al. 

Recent paper by Masci et al.

Identification of graphical model.

Stephane's paper on learning wavelets using pairing.

\section{Generalizing Convolutions in Graphs }

\subsection{Spectral Networks}

Our work builds upon ~\cite{spectralnet2013} which defines spectral networks. We recall the definition here and its main properties.

A spectral network generalizes a convolutional network through the Graph Fourier Transform, which is in turn defined via a generalization of the Laplacian operator on the grid to the graph Laplacian. An input vector $x \in \mathbb{R}^N$ is seen as a a signal defined on a graph $G$ with $N$ nodes. 
\begin{graphconv}
 Let $W$ be a $N \times N$ similarity matrix representing a graph $G$, and let $L= D - W$ be its graph Laplacian with $D=W {\bf 1}$, with eigenvectors $U=(u_1,\dots,u_N)$. Then a \textit{graph convolution} of input signals $x$ with filters $g$ on $G$ is defined by $x \ast_G g = U^T \left( Ux \odot Ug \right)$, where $\odot$ represents a point-wise product. 
\end{graphconv}

Here, the unitary matrix $U$ plays the role of the Fourier Transform in $\mathbb{R}^d$. 
There are several ways of computing the graph Laplacian $L$ \cite{spectralgraphcite}. In this paper, we choose the normalized version $L = I - D^{-1/2}WD^{-1/2}$, where $D$ is a diagonal matrix with entries $D_{ii} = \sum_j W_{ij}$. Note that in the case where $W$ represents the lattice, from the definition of $L$ we recover the discrete Laplacian operator $\Delta$. Also note that the Laplacian commutes with the translation operator, which is diagonalized in the Fourier basis. 
It follows that the eigenvectors of $\Delta$ are given by the Discrete Fourier Transform (DFT) matrix. 
We then recover a classical convolution operator that noting that convolutions are by definition linear operators that diagonalize in the Fourier domain (also known as the Convolution Theorem \cite{mallat}).

Learning filters in a Graph thus amounts to learning spectral multipliers $w_g = (w_1, \dots,w_N)$ 
$$x \ast_G g := U^T ( \mbox{diag}(w_g) U x)~.$$
Extending the convolution to inputs $x$ with multiple input channels is straightforward. If $x$ is a signal with $M$ input channels and $N$ locations, we apply the transformation $U$ on each channel, and then use multipliers $w_g =  (w_{i,j}\, ;\, i \leq N~, j \leq M)$. 

However, for each feature map $g$ we need convolutional kernels are typically restricted to have small spatial support, independent of the number of input pixels $N$, which enables the model to learn a number of parameters independent of $N$. In order to recover a similar learning complexity in the spectral domain, it is thus necessary to restrict the class of spectral multipliers to those corresponding to localized filters. 

For that purpose, we seek to express spatial localization of filters in terms of their spectral multipliers. In the grid, smoothness in the frequency domain corresponds to the spatial decay, since
$$\left| \frac{\partial^k \hat{x}(\xi)}{\partial \xi^k} \right| \leq C \int u^k |x(u)| du~,$$
where $\hat{x}(\xi)$ is the Fourier transform of $x$.
In \cite{spectralnet2013} it was suggested to use the same principle in a general graph, by considering a smoothing kernel $\mathcal{K} \in \mathbb{R}^{N \times N_0}$, such as splines, and searching for spectral multipliers of the form
$$w_g = \mathcal{K} \tilde{w}_g~.$$

The algorithm which implements the graph convolution is described in \ref{pseudoPSO}.

\begin{algorithm}
\caption{Train Graph Convolution Layer}
\label{pseudoPSO}
\begin{algorithmic}[1]
\State Given GFT matrix $U$, interpolation kernel $K$, weights $w$. 
\State \textbf{Forward Pass:}
  \State Fetch input batch $x$ and gradients w.r.t outputs $\nabla y$.
  \State Compute interpolated weights: $w_{f'f} = K \tilde{w_{f'f}}$.
  \State Compute output: $y_{sf'} = U^T\left(\sum_{f} Ux_{sf} \odot w_{f'f} \right)$.
  \State \textbf{Backward Pass:}
  \State Compute gradient w.r.t input: $\nabla x_{sf} = U^T\left(\sum_{f'} \nabla y_{sf'} \odot w_{f'f} \right)$
  \State Compute gradient w.r.t interpolated weights: $\nabla w_{f'f} = U^T\left(\sum_s \nabla y_{sf'} \odot x_{sf} \right)$
  \State Compute gradient w.r.t weights $\nabla \tilde{w_{f'f}} = K^T \nabla w_{f'f}$.
\end{algorithmic}
\end{algorithm}

\subsection{Pooling with Hierarchical Graph Clustering}

In image and speech applications, and in order to reduce the complexity of the model, it is often useful to trade-off spatial resolution with feature resolution as the representation becomes deeper. For that purpose, pooling layers compute statistics in local neighborhoods, such as the average amplitude, energy or maximum activation.

The same layers can be defined in a Graph by providing the equivalent notion of neighborhood. In this work, we construct such neighborhoods at different scales using multi-resolution spectral clustering \cite{spectralclustering}, and consider both average and max-pooling as in standard convolutional network architectures.

\section{ Graph Construction }

\subsection{ Using Prior Knowledge}

Whereas some recognition tasks, such as those considered in \cite{spectralnet2013}, might have a prior knowledge of the graph structure of the input data, many other real-world applications do not have such knowledge. It is thus necessary to estimate a similarity matrix $W$ from the data before constructing the spectral network. 

Word datasets. Using Word embeddings is also a possibility.

We consider in this paper two possible graph constructions.


\subsection{ Estimation from Feature Correlation }

Given data $X \in \mathbb{R}^{N \times L}$, where $L$ is the number of samples and $N$ the number of features,
the simplest approach to estimating a graph structure from the data is to consider a distance between features $i$ and $j$ given by
$$d(i,j) = \| X_i - X_j \|^2~,$$
where $X_i$ is the $i$-th row of $X$. 

This distance is then used to build a Gaussian diffusion Kernel \cite{diffusion_maps} 
\begin{equation}
\label{unsupervisedkernel}
\omega(i,j) = \exp^{-\frac{d(i,j)}{\sigma^2}}~.
\end{equation}
In our experiments, consider the variant of self-tuning diffusion kernel \cite{self-tuning}
$$\omega(i,j) = \exp^{-\frac{d(i,j)}{\sigma_i \sigma_j}}~,$$
where $\sigma_i$ is computed as the distance $d(i,i_{k})$ corresponding to the $k$-th nearest neighbor $i_{k}$ of feature $i$. 

%Pros and Cons
%Semi-supervised Setting: this 
The main advantage of (\ref{unsupervised_kernel}) is that it does not require labeled data. Therefore, it is possible to estimate 
the similarity using several datasets that share the same features, for example in text classification. 
While correlations are typically sufficient to reveal the intrinsic geometrical structure of images \cite{bengioleroux}, 
 the effects of higher-order statistics might be non-negligible in other contexts, especially in presence of sparsity. 

Another strategy is to use $\ell_1$ penalized logistic regression \cite{martin}. 

\subsection{ Supervised Graph Estimation }

Use a fully connected network to determine the feature similarity. 


\section{ Experiments }


\section{Discussion}

The graph construction using the Supervised Graph Estimation. 
Related to discovering latent graphical models


\bibliography{references}{}
\bibliographystyle{plain}

\end{document}
